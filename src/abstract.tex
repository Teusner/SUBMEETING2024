% LaTeX template for submitting an abstract to  
%
%           SWIM 2023
%           July 11-12-13, 2023
%           Angers, FRANCE
%
%% Please use the least number of macros and packages as possible
%% which will help us parsing your abstracts. Thank you! Additionally, 
%% please use UTF8 for font encoding!!! The last but not least, please 
%% do not change the font size and the paper format. 
%%
%% Thank you! We are looking forward for your contributions.
%%
%% Many thanks to all previous SWIM organizers for allowing us to use
%% their template as basis for our style file.

\documentclass[14pt, a4paper]{article}

%% Please use only the following packages. Thank you!
\usepackage{extsizes}
\usepackage{amsmath}
\usepackage{amsthm}
\usepackage{amssymb}
\usepackage{url}
\usepackage{graphicx}
% please use UTF8 for the encoding!!!
\usepackage[utf8]{inputenc}

\pagenumbering{arabic}
\pagestyle{myheadings}
\markright{LARIS/Polytech Angers \hfill SWIM 2023\hfill }
\clearpage

% DEFINITIONS
\newtheorem{prop}{Proposition}

% USER PACKAGES
%\usepackage{solvehaltingproblem}
\usepackage{pgf}
\usepackage{subcaption}

% AUTHORS - if all affiliations are the same, upper indices can be excluded
\newcommand\authors{Quentin Brateau$^{1}$, Fabrice Le Bars $^{1}$ and Luc Jaulin$^{1}$}

% TITLE - required 
\newcommand\papertitle{Union of adjacent contractors}

\begin{document}

	\begin{center}

	% BEGIN DO NOT MODIFY 
	\section*{\papertitle}
	% #REPLACE FOR MENU#
	\vspace*{0.8cm}
	{\large \authors}
	% END DO NOT MODIFY

	\bigskip

	% AFFILIATION - required with address, email at least for corresponding author
	{
		\small $^{1}$ENSTA Bretagne, UMR 6285, Lab-STICC, IAO, ROBEX\\
		2 rue François Verny, 29806 Brest CEDEX 09, \textsc{France} \\
		\medskip
		\texttt{quentin.brateau@ensta-bretagne.org}\\
		\texttt{fabrice.le\_bars@ensta-bretagne.org}\\
		\texttt{lucjaulin@gmail.com}\\
	}

	\end{center}

	\bigskip

	% KEYWORDS - required
	{\noindent\bf Keywords:} Set union, Contractor programming, Geometric contractors, Localization

	\subsection*{Introduction}
		Set theory provides a fundamental structure for interval analysis, which must conform to its formalism~[1,2]. Trivial operations are defined on sets, such as union, intersection, deprivation, cartesian product, and projection. They should be applicable to intervals and therefore to contractors.

		In the case of union of adjacent contractors, typical paving algorithm bisects unnecessarily the boxes and then reveals the common boundary between the two sets as shown in Figure~\ref{fig:sepvisible}. This behavior noticed on contractors union is not consistent with the set union as defined in set theory.

	\subsection*{Geometric contractors}
		Geometric contractors are a class of contractors based on geometric constraints, particularly used for localization in robotics~[2]. They can be used to characterize all possible robot states based on measurements. Geometric contractors are often defined for segments, which are one of the most simple geometric shapes. Then, by using set operators, more complex contractors can be built, such as contractors based on polygons~[3].

		By defining more complex contractors in this way, adjacent boundary-overlapping sets appear at each vertex. Figure~\ref{fig:sepvisible} shows the paving of a visibility separator from a point, implemented by Rémy Guyonneau~[3]. The visibility separator works well on two individual segments but fails to characterize inner subpaving when dealing with polygons.

		\begin{figure}[!htb]
			\centering
			\begin{subfigure}[t]{.31\textwidth}
				\input{imgs/sepvisible_2.pgf}
				\subcaption{Visibility separator with a segment obstacle}
			\end{subfigure}%
			\hfill
			\begin{subfigure}[t]{.31\textwidth}
				\input{imgs/sepvisible_1.pgf}
				\subcaption{Visibility separator with a segment obstacle}
			\end{subfigure}%
			\hfill
			\begin{subfigure}[t]{.31\textwidth}
				\input{imgs/sepvisible.pgf}
				\subcaption{Visibility separator with a polygon obstacle}
			\end{subfigure}
			\caption{Visibility separator for a point relative to obstacles. In red the set visible from the point, in blue the invisible set, in yellow the uncertain set.}
			\label{fig:sepvisible}
		\end{figure}

	\subsection*{Main results}
		Solutions can be found for geometric contractors to prevent the apparition of this common boundary. These solutions rely on contractor-specific solutions to avoid the use of contractors union when building more complex contractors, and some of these will be presented.
		
		However, this problem lays the foundations for a larger issue of the characterization of the union of two adjacent contractors in the general case, for which a solution has not yet been found. Figure~\ref{fig:general_case} shows a paving of the union of adjacent contractors on rings, on which the common boundary is appearing.

		\begin{figure}[!htb]
			\centering
			\begin{subfigure}[t]{.48\textwidth}
				\input{imgs/circle_discontinuity.pgf}
				\subcaption{Union of three adjacent ring separators}
			\end{subfigure}%
			\hfill
			\begin{subfigure}[t]{.48\textwidth}
				\input{imgs/circle_expected.pgf}
				\subcaption{Expected union of three adjacent ring separators}
			\end{subfigure}
			\caption{Union of adjacent boundary-overlapping separators. In red the inner set, in blue the outer set, in yellow the uncertain set.}
			\label{fig:general_case}
		\end{figure}

	% REFERENCES
	\subsection*{References}
		\begin{description}

			\item[{[1]}] L. Jaulin, M. Kieffer, O. Didrit, and E. Walter, Applied Interval Analysis: With Examples in Parameter and State Estimation, Robust Control and Robotics. Springer Science \& Business Media, 2012.
			
			\item[{[2]}] L. Jaulin and B. Desrochers, “Introduction to the algebra of separators with application to path planning,” Engineering Applications of Artificial Intelligence, vol. 33, pp. 141–147, Aug. 2014
			% , doi: 10.1016/j.engappai.2014.04.010.
			
			\item[{[3]}] R. Guyonneau, “Méthodes ensemblistes pour la localisation en robotique mobile,” These de doctorat, Angers, 2013.
			% Accessed: Jun. 06, 2023. [Online]. Available: \url{https://www.theses.fr/2013ANGE0046}
				
	\end{description}

\end{document}
        

%%% Local Variables:
%%% mode: latex
%%% TeX-master: t
%%% End:
